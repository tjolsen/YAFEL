\section{Compilation and Installation}
Upon cloning the YAFEL repository, simply run the ``\texttt{compile.sh}'' script.
This script simply appends a few lines to the end of your .bashrc file
and exports an environment variable YAFELDIR, which is necessary to
compile the library and is useful when linking against the library.
Compiling the YAFEL library is trivially easy on Linux and Unix-like systems.
The library is currently designed to compile using the GNU Make utility.
There are a variety of parameters that can be modified, such as whether 
to parallelize some linear algebra subroutines with openMP.
These parameters are in the file ``\texttt{common.mk}'' in the YAFEL root directory.
